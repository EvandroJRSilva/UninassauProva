\documentclass[a4paper, 11pt, addpoints]{exam}

% A documentação completa da classe Exam pode ser encontrada em:
%		https://ctan.org/pkg/exam
% É bastante recomendável sua leitura, principalmente no caso de você querer modificar alguma configuração.

% Utilizando o estilo de prova da Uninassau
\usepackage{UninassauProva}

% Informações gerais sobre a prova. Vá a esse arquivo para modificar as informações de acordo com a sua necessidade.
%===============================================================================
% Cabeçalho e Rodapé -----------------------------------------------------------
%===============================================================================
\pagestyle{headandfoot}
\firstpageheader{}{}{}
\runningheader{}{}{}
\runningheadrule

\firstpagefooter{\copyright \professor}{}{Pag. \thepage\ de \numpages}
\runningfooter{\professor}{}{Pag. \thepage\ de \numpages}
\runningfootrule

% Começo do Documento
\begin{document}
	% Inserindo o cabeçalho
	\cabecalho
	% Inserindo as informações da prova e garantindo que estarão centralizados na página.
	\info
	
	% Descomente a linha seguinte caso você queira gerar um PDF com as soluções às questões.
	\printanswers
	
	%---------------------------------------------------------------------------
	% QUESTÕES -----------------------------------------------------------------
	\begin{questions} % Ambiente de questões
		% Comando para deixar os textos justificados. É necessário repetir o comando para cada novo ambiente.
		\justifying
		% Cada novo comando \question iniciará uma nova questão. A numeração será adicionada automaticamente.
		\question[1] % O número dentro dos colchetes é o valor da questão.
		Este é o exemplo de uma questão: o que significa o valor entre colchetes?
		%-----------------------------------------------------------------------
		\question[1\half] % Não é possível uma questão ter pontuação "quebrada", ou seja, 1,5. Mas é possível expressar com frações. Outra solução é expressar a pontuação entre 0 e 100.
		Este é o exemplo de uma questão contendo equações ou fórmulas:\\ 
		$x^{2} + 2\pi + \pi^{2}$
		%-----------------------------------------------------------------------
		\question % Definir a pontuação é opcional. Ao se definir junto do comando \question o LaTeX faz cálculos para mostrar a quantidade total de pontos da prova. Não há necessidade desse cálculo, portanto, caso o professor queira, talvez seja melhor acrescentar a pontuação da forma que melhor lhe convir.
		Mais uma questão de exemplo: a definição da pontuação é opcional [1,0].
		%-----------------------------------------------------------------------
		\question
		Uma questão pode também ser dividida em partes listadas como letras.
			% Começando o ambiente de partes
			\begin{parts}
				\justifying % Tem de ser declarado em todo novo ambiente
				\part[1] Essa é a primeira letra. % Uma parte pode também ter pontuação
				\part 
				Fazendo essa questão encontrará a resposta? % Pode ficar na linha seguinte, que nem as questões
				\part Na verdade a resposta você encontra ao resolver a letra (\ref{letraE}).
				\part Resolva a (\ref{letraE}).
				\part Aqui você encontra a resposta.\label{letraE}
					\begin{subparts}
						\subpart Ou será que não? % Subpartes, e subsubpartes são permitidos.
					\end{subparts}
			\end{parts}
		%-----------------------------------------------------------------------
		\question
		Deixando um espaço para que os alunos possam resolver a questão.
		\vspace{3cm} % v[ertical]space
		% Caso você queira distribuir os espaços igualmente enter todas as questões, e partes, subpartes e subsubpartes, basta escrever o seguinte comando ao fim da questão:
		%			\vspace{\stretch{1}}
		% Se alguma questão necessita do dobro de espaço em relação às demais, basta escrever ao fim da questão o seguinte comando:
		%			\vspace{\stretch{2}}
		% O mesmo raciocínio se aplica ao triplo do espaço, etc.
		%-----------------------------------------------------------------------
		\question
		A questão seguinte vem depois do espaço dado para a questão anterior.
			\begin{parts}
				\part O espaçamento pode ser usado nas partes também. \vspace{1cm}
				\part A parte seguinte vem depois de um espaço.
			\end{parts}
		%-----------------------------------------------------------------------
		% Caso queira mudar logo de página, garantindo o espaço restante para a última questão, basta utilizar o comando a seguir:
		\newpage
		\question
		Questão em uma nova página. E dá para deixar linhas para os alunos responderem.
		\fillwithlines{5cm}
		%-----------------------------------------------------------------------
		\question
		Uma questão de múltipla escolha:
			\begin{choices}
				\choice Essa é a alternativa correta.
				\choice Na verdade é essa.
				\choice Não é a anterior.
				\choice Mas talvez seja.
				\choice Melhor marcar essa aqui.
			\end{choices}
		%-----------------------------------------------------------------------
		\question
		Se as alternativas puderem ser em somente uma linha, basta utilizar o ambiente \texttt{oneparchoices}.
			\begin{oneparchoices}
				\choice Eu
				\choice Ele
				\choice Ela
				\choice Nós
				\choice Ninguém
			\end{oneparchoices}
		%-----------------------------------------------------------------------
		\question
		Mas também é possível fazer assim:
			\begin{enumerate}[a)]
				\item Primeira alternativa.
				\item Segunda alternativa.
				\item Terceira alternativa.
				\item Quarta alternativa.
				\item Quinta alternativa.
			\end{enumerate}
		%-----------------------------------------------------------------------
		\question
		Utilizando \texttt{multicols} para separar as alternativas em uma mesma linha:
			\begin{multicols}{5}
				\begin{enumerate}[(A)]
					\item Alternativa 1
					\item Alternativa 2
					\item Alternativa 3
					\item Alternativa 4
					\item Alternativa 5
				\end{enumerate}
			\end{multicols}
		%-----------------------------------------------------------------------
		\question
		Uma questão de múltipla escolha pode ter uma alternativa já marcada como a correta:
			\begin{choices}
				\justifying
				\CorrectChoice A alternativa correta vai aparecer se a opção \texttt{printanswers} estiver ativa no início do documento.
				\choice Alternativa errada.
				\choice Alternativa errada.
				\choice Alternativa errada.
				\choice Alternativa errada.
			\end{choices}
		%-----------------------------------------------------------------------
		\question
		\justifying
		Uma questão pode ter outros tipos de solução (no arquivo UninassauProva.sty tem um trecho onde é possível fazer outras configurações no ambiente de solução):
			\begin{solution}
				\justifying Solução utilizando o ambiente \texttt{solution}. Se \texttt{printanswers} estiver desabilitado, você deve fornecer o tamanho do espaço em branco que será posto no lugar da solução. Caso esse valor não seja especificado, o LaTeX não acrescentará espaço em branco.\\ 
                Ex.: \verb*|\begin{solution}[5cm]|
			\end{solution}
			\begin{solutionorbox}
				Solução com o ambiente \texttt{solutionorbox}. Se a solução não for impressa, haverá no lugar uma caixa de texto em branco, do tamanho especificado pelo usuário.\\ 
                Ex.: \verb*|\begin{solutionorbox}[2cm]|
			\end{solutionorbox}
			\begin{solutionorlines}
				Solução com o ambiente \texttt{solutionorlines}. Se a solução não for impressa, haverá no lugar linhas para o aluno escrever.\\
                Ex.: \verb*|\begin{solutionorlines}[2cm]|
			\end{solutionorlines}
			\begin{solutionordottedlines}
				Solução com o ambiente \texttt{solutionordottedlines}. Se a solução não for impressa, haverá no lugar linhas pontilhadas para o aluno escrever.\\
                Ex.: \verb*|\begin{solutionordottedlines}[2cm]|
			\end{solutionordottedlines}
			\begin{solutionorgrid}
				Solução com o ambiente \texttt{solutionorgrid}. Se a solução não for impressa, haverá no lugar uma grade.\\
                Ex.: \verb*|\begin{solutionorgrid}[2cm]|
			\end{solutionorgrid}
		%-----------------------------------------------------------------------
		% A linha abaixo serve para formatar a aparência da questão bônus.
		\bonusqformat{\textbf{Questão bônus} \dotfill [\thepoints]}
		\bonusquestion[1]
		É possível acrescentar também questões bônus.
	\end{questions}
	
\end{document}
