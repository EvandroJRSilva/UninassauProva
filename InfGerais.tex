%===============================================================================
% Cabeçalho e Rodapé -----------------------------------------------------------
%===============================================================================
\pagestyle{headandfoot}
\firstpageheader{}{}{}
\runningheader{}{}{}
\runningheadrule

\firstpagefooter{\professor}{\disciplina}{Pág. \thepage\ de \numpages}
\runningfooter{\professor}{\disciplina}{Pág. \thepage\ de \numpages}
\runningfootrule


%===============================================================================
% Informações sobre a Avaliação ------------------------------------------------
%===============================================================================
\LogoInst{\includegraphics[width=\linewidth]{Imagens/Logos/logo_uninassau.png}}
\LogoSer{\includegraphics[width=\linewidth]{Imagens/Logos/logo_ser.png}}

% Insira o nome do campus a seguir. Ex.: Teresina.
\Campus{Teresina}

% Insira o nome do curso a seguir. Ex.: Bacharelado em Direito
\NomeCurso{Análise e Desenvolvimento de Sistemas}

% Insira o nome do(a) Professor(a). Ex.: Fulano da Silva
\NomeProfessor{Evandro J.R. Silva}

% Insira o nome da Disciplina. Ex.: Citologia
\NomeDisciplina{Matemática Aplicada}

% Insira a data da prova. Ex.: 10/10/24
\DataProva{04/06/2024}

% Insira a turma. Ex.: FAP0241124NMA
\Turma{FAP0190101GMA}

% Insira o Código da Turma. Exemplo: A
\CodTurma{A}

% Insira o tipo de prova. Ex.: 01, Tipo 01.
\TipoProva{01}

% Insira a prova. Ex.: 1ª, 2ª, Final
\QualProva{2ª}


%===============================================================================
% Instruções para a Avaliação --------------------------------------------------
%===============================================================================

% OBS.: deixe em branco as que não quiser colocar

\instum{A avaliação somente poderá ser entregue depois de decorridos 50 min de seu início.}
 
\instdois{Caneta esferográfica azul ou preta. Provas entregues escritas a lápis \textbf{NÃO} serão corrigidas.}

\insttres{Será atribuída nota zero a aluno que devolver sua prova em branco, independentemente de ter assinado a Ata de Prova.}

\instquatro{Ao aluno flagrado \textbf{utilizando meios ilícitos ou não autorizados pelo professor para responder a avaliação} será atribuída nota zero e, mediante representação do professor, responderá a Procedimento Administrativo Disciplinar, com base
	no Código de Ética.}

%\instcinco{}

%\instseis{}