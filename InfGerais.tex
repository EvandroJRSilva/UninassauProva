%===============================================================================
% Cabeçalho e Rodapé -----------------------------------------------------------
%===============================================================================

% Estilo da página configurado para ter Cabeçalho e Rodapé. Para mais opções verifique a documentação das classe em: https://ctan.org/pkg/exam
\pagestyle{headandfoot}
% Cada {} corresponde à esquerda, centro e direita do Cabeçalho ou Rodapé
\firstpageheader{}{}{}
\runningheader{}{\cabecalho}{} 
\extraheadheight[0cm]{1.5cm} % altura extra de cabeçalho (exceto na primeira página) para o conteúdo aparecer completo nas páginas seguintes.
\runningheadrule

\firstpagefooter{\professor}{}{\disciplina}
\firstpagefootrule
\runningfooter{\professor}{\disciplina}{Pág. \thepage\ de \numpages}
\runningfootrule


%===============================================================================
% Informações sobre a Avaliação ------------------------------------------------
%===============================================================================
\LogoInst{\includegraphics[width=\linewidth]{Imagens/Logos/logo_uninassau.png}}
\LogoSer{\includegraphics[width=\linewidth]{Imagens/Logos/logo_ser.png}}

% Insira o nome do campus a seguir. Ex.: Teresina.
\Campus{Teresina}

% Insira o nome do curso a seguir. Ex.: Bacharelado em Direito
\NomeCurso{Nome do Curso}

% Insira o nome do(a) Professor(a). Ex.: Fulano da Silva
\NomeProfessor{Nome do(a) Professor(a)}

% Insira o nome da Disciplina. Ex.: Citologia
\NomeDisciplina{Nome da Disciplina}

% Insira a data da prova. Ex.: 10/10/24
\DataProva{04/06/2024}

% Insira a turma. Ex.: FAP0241124NMA
\Turma{FAP0101010GMA}

% Insira o Código da Turma. Exemplo: A
\CodTurma{A}

% Insira o tipo de prova. Ex.: 01, Tipo 01.
\TipoProva{}

% Insira a prova. Ex.: 1ª, 2ª
\QualProva{1ª}

% Configurando a prova como de primeira chamada por padrão
\newbool{SegundaChamada} % Quando criado é false por padrão
% Descomentar a linha abaixo, caso a prova seja de Segunda Chamada
%\booltrue{SegundaChamada}

% Configurando a possibilidade de ser Prova Final
\newbool{ProvaFinal} % Quando criado é false por padrão
% Descomentar a linha seguinte caso seja a Prova Final
%\booltrue{ProvaFinal}

%===============================================================================
% Instruções para a Avaliação --------------------------------------------------
%===============================================================================

% OBS.: deixe em branco as que não quiser colocar. Mas caso queira cinco ou mais instruções, descomente \instcinco{} em diante, e utilize novos comandos caso seja necessário. No arquivo UninassauProva.sty procure pelos trechos em que os comandos são criados, como também o trecho em que os comandos são mostrados na prova.

\instum{A avaliação somente poderá ser entregue depois de decorridos 50 min de seu início.}
 
\instdois{Caneta esferográfica azul ou preta. Provas entregues escritas a lápis \textbf{NÃO} serão corrigidas.}

\insttres{Será atribuída nota zero a aluno que devolver sua prova em branco, independentemente de ter assinado a Ata de Prova.}

\instquatro{Ao aluno flagrado \textbf{utilizando meios ilícitos ou não autorizados pelo professor para responder a avaliação} será atribuída nota zero e, mediante representação do professor, responderá a Procedimento Administrativo Disciplinar, com base
	no Código de Ética.}

%\instcinco{}

%\instseis{}
